\documentclass[12pt]{article}

\usepackage{minted}
\usepackage{hyperref}
\usepackage{datetime}
\usepackage{datenumber}
\usepackage{advdate}
\usepackage[super]{nth}
\usepackage[margin=0.75in]{geometry}

\parindent 0pt \parskip 6pt

\newcommand\todo[1]{\textbf{TODO(kc506): #1}}
\newcommand\haskell[1]{\mintinline{haskell}{#1}}
\newcommand\monospace[1]{\mintinline{text}{#1}}

\begin{document}

\thispagestyle{empty}

\centerline{\large Computer Science Tripos: Part II Project Proposal}
\vspace{0.4in}
\centerline{\Large\bf An Optimising Compiler from Haskell to Java Bytecode}
\vspace{0.3in}

\centerline{Keith Collister, kc506}
\centerline{Robinson College}

\centerline{\large \emph{date}}

\vspace{1in}

\begin{tabular}{ p{4cm} p{4.5cm} l }
{\bf Project Originator:} & Keith Collister & \\[3mm]
{\bf Project Supervisor:} & Dr. Timothy Jones & {\bf Signature:} \\[3mm]
{\bf Director of Studies:} & Prof. Alan Mycroft & {\bf Signature:} \\[3mm]
{\bf Overseers:} & Andrew Rice & {\bf Signature:} \\[3mm]
                 & Prof. Simone Teufel & {\bf Signature:} \\[3mm]
\end{tabular}

\todo{Andy Rice title - Dr/Prof/...?}

\vspace{0.75in}

% https://www.cl.cam.ac.uk/teaching/projects/modelprojs/


\section*{Introduction and Description of the Work}

The goal of this project is to implement an optimising compiler from a subset of the Haskell language to Java bytecode.
A variety of optimisations will be implemented to explore their effect on compilation and execution time, as well as on
the size of the output bytecode.

Haskell is a functional, pure, and non-strict language seeing increasing usage in industry and academia. Purity makes
programs much simpler to reason about: a programmer can usually tell from the type of a function exactly what it can do,
which makes it easier to avoid bugs.

Java Bytecode was chosen as the target language as it is portable and mature. While not as performant as native machine
code, bytecode output by the compiler built during this project can be interpreted on almost any platform, rather than
being restricted to e.g.\ only machines with an x86-64 processor.


\section*{Starting Point}

I intend to use Haskell to develop the compiler, and Python or Bash for quick utility scripts -- I have experience with
all of these languages.

I've preread the 2018 Optimising Compilers course\footnote{https://www.cl.cam.ac.uk/teaching/1718/OptComp/} as
preparation: my schedule involves writing optimisations before the module is lectured.


\section*{Resources Required}

I will use my personal laptop to develop this project: a ThinkPad 13 running NixOS. I will use Git for version control,
host the code on a public repository on GitHub, and use TravisCI for automated tests. I also intend to keep a backup
repository on an MCS machine -- my personal DS-Filestore allowance should be sufficient.

Should my laptop break or otherwise become unusable to complete the project, I have an older laptop running Debian 9
that I can use. It should only cost a few days to get it set up with a Haskell development environment.

I intend to use the GHC compiler\footnote{https://www.haskell.org/ghc/} with the Stack
toolchain\footnote{https://github.com/commercialhaskell/stack} for development (both are available under BSD-style
licenses).


\section*{Substance and Structure of the Project}

The aim of the project is to develop an optimising compiler that can translate simple programs written in Haskell into
Java bytecode that can be interpreted on platforms supporting the Java Runtime Environment.

As the focus of the project is on the implementation of various language features and optimisations operating on them, I
intend to use an existing library for lexing and parsing
(\monospace{haskell-src}\footnote{https://hackage.haskell.org/package/haskell-src}) which can produce an AST from
Haskell 98. This will allow for more time to be devoted to those parts of the project which are more aligned with the
aim.

Similarly, I intend to use the \monospace{hs-java} library\footnote{https://hackage.haskell.org/package/hs-java} to
handle the conversion from the logical representation of bytecode to the actual instruction sequences (the ``assembly''
part of the backend).

As Haskell is a lazy language, one of the major challenges will be to design a way to represent and perform lazy
computation. This might be achieved using ``thunks'', in-memory representations of pending computations. GHC uses a
directed graph to keep track of thunks.

Haskell is a very feature-rich language, and those features are often highly dependent on each other: simple things
often touch many aspects of the language (for example, the simple numeric literal \haskell{5} which would have type
\mintinline{c}{int} in C instead has type \haskell{Num t => t} in Haskell, involving type classes and type constraints.
Simple operations like addition and subtraction on integers are provided by \haskell{Int}'s \haskell{Num} instance). In
the schedule below I've detailed which areas I intend to focus on for language features -- the subset should hopefully
be small enough to be implementable in reasonable time, but large enough to allow for interesting programs to be
written.

The project also aims to implement some optimisations, to improve the performance of the output of the compiler. These
are also detailed in the schedule below, and include a mix of classical optimisations (e.g.\ peephole) and more exotic
ones (strictness analysis).

Tests are a vital part of any engineering project. During the work on the project, I will write and maintain a test suite
to ensure the various components of the compiler work as expected and guard against regressions. I intend to use the
\monospace{tasty} framework\footnote{https://hackage.haskell.org/package/tasty} which provides a standard interface to
\monospace{HUnit}\footnote{https://hackage.haskell.org/package/HUnit} (for unit and regression tests) and
\monospace{QuickCheck}\footnote{https://hackage.haskell.org/package/QuickCheck} (for wonderful property-based tests) to
implement this test suite.


\section*{Success Criteria}

The primary goal of the project is to produce a compiler that can translate source code written in a small subset of
Haskell into Java bytecode suitable for execution by the JVM, performing simple optimisations during the translation
process. This big-picture goal can be subdivided into distinct aspects:

\begin{itemize}
\item
{
    To succeed as a Haskell compiler, the result of this project should be able to reject ill-formed programs for
    syntactic or type errors (within the scope of the subset of Haskell implemented), and convert well-formed programs
    into Java bytecode. The output programs should perform computation non-strictly.
}
\item
{
    To succeed as an optimising compiler, it should produce bytecode which executes faster and/or in less memory than
    bytecode produced by a non-optimising compiler (this can be tested by enabling/disabling optimisations within this
    compiler and comparing the relative execution speeds/memory footprint).
}
\item
{
    To succeed as a stable and reliable piece of software, it should have a comprehensive test suite including unit
    tests and regression tests for bugs found during development.
}
\end{itemize}


\section*{Extensions}

There are plenty of interesting extensions to the proposed work:

\begin{itemize}
\item
{
    There are many more features to Haskell than those mentioned in this proposal, ranging from syntactic sugar to
    features in their own right: infix operators, operator sections, point-free notation, user-defined datatypes, type
    instances, monads, do-notation for monads, do-notation for applicative functors, GADTs, user input/output, etc.

    It would be fun to increase the size of the implemented subset of Haskell, both in order to write more interesting
    programs and also to explore the effectiveness of existing optimisations on the new changes.
}
\item
{
    There exist many more optimisations that could be investigated: there are over 60 ``big picture'' optimisations
    listed on the GHC's ``using optimisations''
    page\footnote{https://downloads.haskell.org/~ghc/master/users-guide/using-optimisation.html}.
}
\item
{
    One of the greatest attractions of pure languages is the relative ease with which they can be parallelised: any
    subexpressions can be evaluated at any time without effecting the result of the computation. GHC provides a
    concurrency extension including a number of simple language constructs and runtime environment options that allow
    for parallel computations to be expressed in a clean way. It would be extremely interesting to implement something
    similar (but sadly also very time consuming and far beyond the scope of this project).
}
\item
{
    The Haskell Prelude\footnote{https://www.haskell.org/onlinereport/standard-prelude.html} is the ``standard library''
    of Haskell: as it is written in Haskell, it might be possible to compile parts of it using the compiler developed
    during this project, allowing it to be used in programs. However, this would require quite a significant level of
    support for the language in the compiler.
}
\item
{
    A very cool demonstration for the project would be to compile the project using the compiler developed during the
    project (bootstrapping). This would require extensive language support though (at the very least, support for
    monads), which is likely infeasible to be completed.
}
\item
{
    One potential advantage of using the JVM as a target is that it may be possible to provide an interface between Java
    code and Haskell code.
}
\end{itemize}

\section*{Schedule}

I intend to treat tests as part of a feature: when the schedule lists a certain feature as being deliverable in a slot,
that implicitly includes suitable tests for it.

% Set start date to 15th October 2018
\ThisYear{2018}\ThisMonth{10}\ThisDay{15}

% Specify date format like "15th Oct"
\newdateformat{datefmt}{\nth{\THEDAY} \shortmonthname[\THEMONTH]}

% Force advdate commands to change the global "today" instance, not the local one
\makeatletter
\renewcommand\AdvanceDate[1][\@ne]{\global\advance\day#1 \FixDate}
\renewcommand\FixDate{%
  \FixMonth \is@LeapYear
  \l@@p \global\ifnum\day<1 \Pr@vD@y \repeat
  \l@@p \M@s\m@sic \global\ifnum\day>\M@s \N@xtD@y \repeat
}
\renewcommand\FixMonth{%
  \L@@p \global\ifnum\month<1 \global\advance\year\m@ne \global\advance\month12 \is@LeapYear \repeat
  \L@@p \global\ifnum\month>12 \global\advance\year\@ne \global\advance\month-12 \is@LeapYear \repeat}
\def\Pr@vD@y{%
  \global\ifnum\day<-366
    \global\ifnum\month>2
      \global\advance\day\r@k \global\advance\year\m@ne \is@LeapYear
    \else
      \global\advance\year\m@ne \is@LeapYear \advance\day\r@k
    \fi
  \else
    \global\advance\month\m@ne \FixMonth
    \global\advance\day\m@sic
  \fi}
\def\N@xtD@y{%
  \global\ifnum\day>366
    \global\ifnum\month>2
      \global\advance\year\@ne \is@LeapYear \global\advance\day-\r@k
    \else
      \global\advance\day-\r@k \global\advance\year\@ne \is@LeapYear
    \fi
  \else
    \global\advance\day-\M@s \global\advance\month\@ne \FixMonth
  \fi}
\makeatother

% Output eg. boldface "15th Oct - 29th Oct" when called.
\newcommand{\daterange}[1]{%
    \textbf{\datefmt\today}
    \textbf{--}
    \AdvanceDate[#1]\relax
    \textbf{\datefmt\today}
    % Move one day past the end, intervals are exclusive
    \AdvanceDate[1]\relax
}

\begin{itemize}
\item
{
    \daterange{7}

    General project setup: creating a version-controlled repository of code with continuous integration to run tests.

    Create a simple frontend for converting a given file into an AST using the \monospace{haskell-src} package. Focusing
    on single files for the moment, as without proper support for modules, multiple source files don't make sense.
}
\item
{
    \daterange{21}

    Implement a typechecking pass over the AST, \textit{including support for typeclasses}. This is one of the most
    uncertain duration parts of the project, because while the Hindley-Milner type system is well understood and
    frequently implemented, the extension of type classes seems less comprehensively covered, although there are still
    some strong leads\footnote{http://citeseerx.ist.psu.edu/viewdoc/download?doi=10.1.1.53.3952\&rep=rep1\&type=pdf}.

    After this work, the frontend should be minimally functional and the compiler should be able to reject ill-formed
    source code either due to syntactic or type errors.
}
\item
{
    \daterange{21}

    Create a simple (non-optimising) backend for experimenting with lazy evaluation. This should just perform a
    minimally-featured translation from the frontend's AST to executable bytecode (supporting e.g.\ basic arithmetic and
    conditional expressions), but performing evaluation \textit{lazily}.

    For example, \haskell{1 + if true then 2 else undefined} should evaluate without touching the \haskell{undefined}
    subexpression and \haskell{1 + 2} should generate appropriate bytecode to evaluate the expression only when needed).

    A lot of time has been allocated for this, as I expect that there'll be a lot of edge cases and tricky decisions to
    make when implementing lazy evaluation. Care needs to be taken to ensure that the resulting code really does behave
    non-strictly, and that the implementation doesn't have any major bugs that'll block later work.
}
\item
{
    \daterange{14}

    The goal of this week is to implement a peephole pass to collapse sequences of instructions into more efficient
    versions. The sequences to be collapsed will need to be decided at the time, based on inspection of the bytecode
    output by the compiler, and more peephole rules can be added as other transformations are implemented.
    
    After this work is complete, the absolutely minimal success criteria should have been met, taking pressure off the
    rest of the planned work.
}
\item
{
    \daterange{7}

    This week is a slack week, to catch up on anything that fell behind, or to spend time cleaning up any parts of the
    existing implementation that are messy/fragile/poorly designed.
}
\item
{
    \daterange{21}

    Implement user-defined functions, supporting currying, partial application, recursion, and laziness. Depending on
    how long this takes, this may be a convenient time to implement a number of smaller related features, such as:
    
    \begin{itemize}
    \item Pattern matching.
    \item \haskell{case} expressions.
    \item \haskell{let ... in ...} expressions.
    \item \haskell{... where ...} expressions.
    \end{itemize}
}
\item
{
    \daterange{21}

    Introduce lists: these are one of the most frequently used data structures in Haskell, and form the basis for many
    algorithms. They also give many opportunities to demonstrate that the implementation of lazy evaluation works
    correctly (e.g.\ by careful analysis of expressions like \haskell{let l = 1:l in take 5 l}, which should give
    \haskell{[1,1,1,1,1]}).

    I will also write my progress report and presentation during these weeks.
}
\item
{
    \daterange{14}

    Implement an optimisation pass performing common subexpression elimination. In a lazy language, where we can expect
    to see significant buildup of unevaluated computations in memory, ensuring that we don't have two large blobs of
    memory devoted to computing the same thing is likely worthwhile.

    % Instead of CSE, maybe do inlining as it subsumes CSE and has additional benefits.
    % https://www.microsoft.com/en-us/research/wp-content/uploads/2002/07/inline.pdf
}
\item
{
    \daterange{21}

    Implement strictness analysis, and accompanying optimisations. The optimisation opportunities revealed by
    strictness analysis should reduce compiled code size and memory usage, by eagerly evaluating expressions that are
    guaranteed to require evaluation during program execution.
}
\item
{
    \daterange{28}

    During these weeks, I intend to focus on writing the dissertation.

    Implement some micro-benchmarks to demonstrate the effectiveness of the optimisations, for use in the evaluation.
}
\item
{
    \daterange{14}

    In these weeks I hope to balance work on the dissertation with revision.
}
\item
{
    \daterange{14}

    I now expect to switch fully to revision, making only critical changes to the dissertation.

    At the end of these weeks I hope to submit the dissertation.
}
\end{itemize}

\end{document}